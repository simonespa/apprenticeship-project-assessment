\documentclass[12pt,a4paper]{article}

\usepackage{hyperref}

\begin{document}

% \title{Beyond Metadata for BBC iPlayer:\\An Autoencoder-driven approach for embeddings learning to enhance content similarity recommendations.}
\title{Beyond Metadata for BBC iPlayer:\\an autoencoder-driven approach for embeddings generation in content similarity recommendation}
\author{Simone Spaccarotella}
\date{September 2024}

\maketitle
\tableofcontents

\section{Introduction and background}

% Explain who you are, what your role is and why this project is relevant to yourself, remember the assessor will not have met you before.
% Give some context with relevant information on the organisation.

I am a Software Engineer at the BBC, Team Lead for the Sounds web team, and I have been training as a Data Scientist,
working in attachment with the iPlayer Recommendation team.

I built a model that produces content-to-content (C2C) similarity of Video On Demand, for the "More Like This" \cite{MoreLikeBluey} for BBC iPlayer.
This project is relevant to me because it is about recommendations and I have been crossing paths with this world multiple times during my career at the BBC.
I had a tangent encounter back in 2015, while working for a team that was building
an initial recommender for BBC News and an API to provide recommendations using 3rd party engines.
During a Hack Day, I produced and presented a talk called
"Recommendation Assumptions" \cite{RecsAssumptions}, which was about the external factors affecting recommendations, contextual to the consumption
of the content itself.

The BBC is a well-known British broadcaster, and it is always evolving to remain relevant to its audience. Its mission
is to inform, educate and entertain, and it operates within the boundaries set by the Royal Charter \cite{RoyalCharterBBC}.
The current media landscape requires the BBC to deliver digital-first content that is relevant to the audience,
and this involves investements in data and personalised services, not to mention a certain revolution in machine learning
that is keeping everyone busy.

\section{Outline of the issue or opportunity and the business problem to be solved}

\section{Methods used \& justification}

\section{Scope of the project and Key Performance Indicators}

\section{Data selection, collection \& pre-processing}

\section{Survey of potential alternatives}

\section{Implementation - performance metrics}

\section{Results}

\section{Discussion \& conclusions/recommendations}

\section{Summary of findings}

\section{Implications}

\section{Caveats \& limitations}

\section{Appendices}

\bibliography{refs}
\bibliographystyle{plain}

\end{document}
