% Provide high-level overview of the methods (e.g. regression) and tools used
% with justification explaining why these methods are suitable for the problem at hand.
% Justification may include factors such as the need for interpretability, transparency
% or the timescales you had to work with.

\section{Methods and justification}

\subsection{Data pre-processing}

I used \textbf{one-hot encoding} to transform the categorical features (i.e. the metadata annotations) into a numerical vector.
It's a simple yet effective encoding method and it's perfect for the transformation of nominal categoricals,
because it doesn't introduce any ranking and/or arithmetic relationship among the
encoded values. The downside of this approach is that it generates high-dimensional sparse arrays,
introducing the so called \textit{curse of dimensionality} problem. Nonetheless, this was an accepted drawback that was
managed in the modelling phase.

\subsection{Modeling and regularisation}

I trained an undercomplete \textbf{autoencoder} to improve the the quality of the recommendations, and to reduce the cost of calculating them.
Autoencoders are self-supervised models, capable of capturing non-linearity from the data. This type of encoder-decoder limits
the number of nodes in the hidden layers, and creates a "bottleneck" of information flow through the neural network.
This bottleneck structure is a form of \textit{regularisation} that forces the model to learn latent attributes from the input,
while reconstructing it with minimal loss. Ultimately, it prevents the model from \textit{overfitting} the training dataset,
by indexing it like a caching layer.

The \textit{encoder} part of the trained encoder-decoder is used to compress the one-hot encoded high-dimensional sparse vectors into
so-called \textit{embeddings}, a lower-dimensional dense representation of the initial content metadata.
This technique solved the curse of dimensionality and the data sparsity problems, and improved the calculation complexity and the quality
of the recommendations at inference time.

To improve and assess the ability of the model to \textit{generalise} on unseen data, I randomly shuffled the dataset and split
it into 3 chunks: training, validation and test. I run \textbf{hyperparameters tuning} to find the best set of model parameters
that minimised the objective function and I used the validation set to regularise the model with \textbf{early stopping}.
This technique monitored the reconstruction loss on an out-of-sample dataset,
allowing the model to stop training within a certain "patience" threshold, after reaching a local minima on the validation error.

I used \textbf{dropout} to further regularise the model and make it robust to small changes in the input,
and \textbf{data augmentation}, by including in the training data the episodes of a programme that share the same tags with their
parent container.

\textbf{Weight decay} and \textbf{batch normalisation} were tested during hyperparameter tuning and discarded for poor performances.% Batch norm

\subsection{Inference}

Item similarity was calculated with the \textbf{cosine} of the angle $\theta$ between each pair of embeddings. This metric
is insensitive to the magnitude of the vectors, and because popular values tend to have a larger magnitude, it mitigates the impact of
popularity in the similarity calculation. In addition, multi-hot encoded pairs can only have a finite number of angles.
They represent unit vectors bound in the "positive quadrant" of a multi-dimensional Euclidean space.
This forces the cosine similarity to also assume a finite number of discrete values between 0 and 1, leading to an information
loss. Ideally, we would expect the similarity score to assume a continuous value bound between -1 and 1, and this
is only possible if the angle $\theta$ of any vector pair can assume a continuous value between 0 and 360 (i.e. $0\pi$ and $2\pi$), hence
the use of embeddings.

\subsection{Tools and frameworks}

The entire project was written in \textbf{Python}. It is the \textit{de facto} programming language for data science
and machine learning tasks. Python has an established, diverse and well-documented ecosystem of external libraries
and frameworks that facilitated the job, and it is also the language of choice at the BBC.

I used \textbf{Pandas} only for tabular data manipulation, to generate and store the one-hot encoded vectors. Unfortunately it wasn't possible
to use it for exploratory data analysis (EDA), because the iPlayer catalogue used in development had roughly one-year worth of data and it didn't
fit in memory, causing Pandas to crash. For this reason I used \textbf{Dask}. It is a library capable of running out-of-memory and
parallel execution for faster processing on single-node machines and distributed computing on multi-nodes machines, while using the familiar Pandas API.

I used \textbf{TensorFlow} and \textbf{Keras} for modelling, to build and train the encoder-decoder neural network architecture,
and \textbf{Keras Tuner} for
hyperparameters tuning. I also used \textbf{Scikit-learn} but not for modelling. It provided utility functions for the dataset splitting and
the cosine similarity calculation, and I was already familiar with its API.

For data visualisation I used a combination of \textbf{Matplotlib} and \textbf{Seaborn}, while I used \textbf{rdflib} to fetch and parse
the RDF documents from the BBC Ontology. These documents represented the metadata values and contained the actual entity label.
This label was needed to hydrate the set of metadata for each item, to visualise the recommendations for testing purposes.
Worth also mentioning the use of \textbf{pytest} for unit testing and \textbf{black} for PEP 8 code compliance and formatting.
Finally, I used \textbf{Jupyter Lab} to edit the project, \textbf{git} for code versioning, \textbf{GitHub} as a remote code repository
and for collaboration, and \textbf{AWS Sagemaker} to run the pipeline on more capable virtual machines, especially during hyperparameter tuning.
