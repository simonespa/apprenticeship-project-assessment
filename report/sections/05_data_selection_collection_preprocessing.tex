% This gives a good opportunity to briefly describe the data architecture of your company
% You must show you have considered all relevant data protection policies and regulations (both internal company policy and external laws or guidelines).
% If you use any data external to the company you must reference it.
% Justify the choice of the data and why it is relevant to the project
% State what tools you used to efficiently gather the data
% Outline briefly the data cleansing steps you have taken and the consideration of any source of error and bias

\section{Data selection, collection and pre-processing}

% Data architecture

Recommended programs can be one-offs like movies or documentaries or have episodes like series or live shows,
which can be arranged hierarchically.
A program is mapped internally by a tree-like structure of items defined by an ID called PID and a type.
Each item of the tree represents a part of the program (a single episode, season, or the programme itself)
and shares the same Passport tags by inheritance.
Because of this data structure, each dataset row represents a faction of a recommendable program,
repeating the same set of tags for a given one.

BBC News and Sport articles, iPlayer videos or Sounds audio are annotated with Passport tags.
These tags describe any content produced by the BBC and can be used for retrieval (search) and filtering (recommendations).
They can be applied either manually by an editorial team with domain knowledge,
or semi-automatically by machine learning algorithms with human supervision.

Passport tags are distributed across the BBC via the universal content exposure and delivery (UCED) system.
This self-service metadata delivery platform exposes data as a document stream for products to integrate.
This platform provides different types of consumers, such as REST API or, AWS S3 bucket, and so on.
Passport documents are JSON objects that contain a property called ``\verb|taggings|'', an array of objects representing the
metadata annotations. Two properties describe these objects: ``\verb|predicate|'' and ``\verb|value|''.
They represent a tag's name and value and are expressed as URL-formatted strings, except for dates.
The predicate is a class of the BBC Ontology \cite{BBC:Ontologies},
while the value can be a date or an entity defined as an RDF \cite{W3C:RDF,W3C:RDF:Concepts} document,
accessible in Turtle format \cite{W3C:Turtle} via the BBC Things API \cite{BBC:Things,BBC:Things:About,BBC:Things:API}.
These entities are linked to each other and external resources and are described by attributes and relationships,
giving the data a graph structure.

% Tools used to gather data

I decided not to integrate with UCED during development but to use batches of Passport files,
manually collected and stored in a local folder.
This trade-off allowed me to train the model with live data while keeping costs down.
In addition, because I had to create resources on two AWS accounts,
I didn't want to pass the burden of maintenance to the team that owned them without having tested the feasibility of the solution first.

% Data protection policies and regulation

Content metadata does not constitute personal data and, therefore, is not subject to the UK GDPR  \cite{UKGDPR}.
Nonetheless, this data is encrypted at rest and in transit by default.
For this reason, no further actions were required during storage and processing.

% Data choice justification and relevance

I chose to use Passport because it provides a set of tags shared across all content produced by the BBC,
making this a general solution that reduces duplications and costs.
Passport offers a flexible and rich set of tags to describe any type of content.
Annotations can describe canonical information such as \textit{genre} and \textit{format},
the \textit{contributor} featured in the programme, the \textit{narrative theme}, the \textit{editorial tone},
what the content is \textit{about} or what relevant entities are \textit{mentioned}, and many more.

% Data preprocessing, errors, bias

The pre-processing stage of the pipeline loaded a list of JSON files and extrapolated the tags into a dictionary data structure.
The dictionary's key was the programme ID - known internally as \textit{PID} - and the value was another dictionary describing the annotations.
A program can be tagged with the same predicate multiple times if it has different values,
while the same value (e.g. ``Music'') can be used by various predicates (e.g. \verb|about| or \verb|genre|).

The pipeline then transformed this dictionary data structure into a Pandas \textit{Dataframe},
where the rows represented the programmes, and the columns represented the tags.
I used a MultiIndex \cite{Pandas:MultiIndex} for the columns because I needed to keep the duplicate values (2nd-level index) across the predicates
(1st-level index).
It then populated the cells of the \textit{Dataframe} with the value ``\verb|1|''
if the program was annotated with the corresponding tag or ``\verb|0|'' otherwise, generating one-hot encoded arrays.
I initially adopted a vectorisation approach known to be performant, using the ``\verb|get_dummies|'' Pandas function.
Surprisingly, this method was slower and less scalable than the solution that I adopted in the end.

A source of bias in the dataset was the ``\verb|mentions|'' tag. This annotation type is automatically generated by an algorithm that extracts
terms deemed important, appearing in the text of an article or the transcript of an audio/video content.
If something is ``mentioned'', it doesn't necessarily describe what the content is about because of the intrinsic ambiguities of natural languages.
Figure of speech devices, such as metaphors, analogies, allegories, and others, alter the meaning of a sentence for stylistic effect and can misrepresent the main topic.
For example, if the phrase ``being over the moon'' is mentioned
by someone delighted about something unrelated to the topic of ``space'' and ``universe'',
extracting the term ``moon'' as a descriptor could mislead the representation.
To mitigate this source of bias, I dropped the tag in favour of ``\verb|about|'', another tag that describes what the content is really about.
A team of editorials annotates content with this tag, using relevant topics.

Generating embeddings of one-hot encoded vectors with the same size used in training but with
unseen tags leads to unpredictable errors.
The encoding is positional, and the combination of \verb|1| and \verb|0| learned by the model
belongs to the tags seen during training. So, I decided to drop the new tags and encode only the ones the model was trained on
while padding the rest with zeros, pending retraining to capture the new information.
Also, some programmes didn't have any annotations entirely. Adding them to the training data, created a group of entries with all zeros.
If enough observations shared this uninformative characteristic, the model could have picked it up.
I decided to drop these programmes and include only the ones with at least one annotation.
