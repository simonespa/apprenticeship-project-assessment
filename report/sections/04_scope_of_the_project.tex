% Detail what the minimum desirable outcome is and how success will be measured.

\section{Scope of the project and Key Performance Indicators}

This project's scope was to build an end-to-end machine learning solution that could produce non-personalised content-to-content similarity recommendations using Passport metadata tags as input.

The minimum viable outcome was to produce recommendations comparable to those currently in production,
while the desired outcome was to increase user engagement.
The integration with Passport would make this solution general and applicable to multimodal BBC content.
If adopted by $N$ BBC products with a total cost of $C$,
it could generate considerable savings, with an approximate cost reduction by a factor of $\frac{C}{N}$.

Comparability was a qualitative and subjective key performance indicator (KPI) that served as a compass,
indicating whether the project was progressing towards the right direction.
Technical and non-technical stakeholders with diverse domain knowledge and background were involved.
I built a rudimentary tool that visualised the programme's title, image, description and Passport tags,
comparing the seed programme with the top-K recommendations.
The people involved gave their subjective feedback on their perceived level of similarity of the output,
testing edge cases and sensitive recommendations like content recommendations for children's accounts.
They also discussed anomalies and unexpected results.
Opening the ``More Like This'' tab on any programme page on BBC iPlayer allowed them to compare the live recommendations
with those generated by the pipeline.

The solution needed to be A/B tested in production, with live data, to measure user engagement.
Unfortunately, too many moving parts outside my control were required to happen for me to build a
production-worthy version to achieve that.
Adopting this as a KPI would have delayed the project, increasing the odds of failure.
To mitigate this risk, I had to decouple it from the project's success.

I defined a hypothesis that could be tested offline and within my control.
The hypothesis stated that long-term user engagement is not just about accuracy. It can be affected by increasing diversity in the recommended content.
A diverse set of recommendations generates new and unexpected results, increasing surprise and serendipity, pushing the user away from boredom.
This theory was untested but grounded in active research on the topic, such as \cite{Kaminskas2016DiversitySN} and
\cite{duricic2023beyondaccuracyreviewdiversityserendipity}, which made it less far-fetched.
For this reason, I decided to measure diversity offline based on the frequency and distribution of the tags that annotated the recommendations,
pending future A/B testing to validate the hypothesis.
