% Best practice in any data science problem is to consider a range of appropriate models
% Detail the alternative models
% State the pros and cons of each, including the justification for your chosen approach.
% Critical evaluation of your methods is key

\section{Survey of potential alternatives}

I initially considered using \textbf{clustering} on the one-hot encoded features to group similar items.
The model would have returned the items belonging to the same cluster as the one considered for similarity.
However, I didn't know how many clusters there could be in the data, and I didn't need to.
I could have used a density-based technique to auto-discover them,
but even in that case, some of the clusters could have had less than K items in a top-K similarity scenario.
I could have returned the items belonging to the nearest cluster if needed, but because of this uncertainty,
clustering wasn't very useful in this use case.

The geometric interpretation of similarity between two items, is the distance in space between the two vectors representing them.
Any item has a degree of similarity with all the others, and clustering was just a coarse-grained discretisation of that concept.
I needed a more granular approach, where every item could be compared with anyone else.
In geometric terms, I had to calculate the \textbf{pairwise distance} between all vectors, given a metric.
So, I discarded clustering as a candidate option.

One-hot encoding doesn't use spatial proximity information to transform the categorical features into ones and zeros.
It's a transformation process that pivots the unique values of each original feature to be the new variables of the transformed vector.
If we project these raw vectors in a multidimensional space, we wouldn't be able to use their
relative position to each other as a similarity measure. Moreover, the high dimensionality of the vectors would have increased
the computational complexity.

To calculate the pairwise distance efficiently and produce a meaningful representation of similarity,
the vectors needed to be in a denser and lower dimensional space,
a manifold embedded into the original high-dimensional ambient space.

I considered dimensionality reduction techniques such as \textbf{principal component analysis (PCA)},
\textbf{independent component analysis (ICA)} or \textbf{linear discriminant analysis (LDA)},
but there was a problem with them too.
Their job is to find a linear projection of the data,
but this is a strong assumption that misses important non-linear structures.
I didn't use PCA (which I was more familiar with),
but I didn't discard the idea of using dimensionality reduction.
I just needed a non-linear approach and turned my attention to manifold learning.

Before introducing the chosen approach in the next section and discussing the pros and cons,
I'd like to describe the alternative pre-processing step I also considered to generate vectors
that didn't suffer from the curse of dimensionality.

\textbf{Hashing} is a non-invertible transformation that can be used for feature reduction.
It can generate smaller vectors than one-hot encoding, but I didn't adopt it
because hashing has more hyperparameters, which increases its complexity.
Most importantly, it introduces the \textit{collision} problem,
where two distinct inputs can be mapped to the same index in the same target domain.
This issue could be mitigated by choosing the latest and most robust algorithm to reduce the likelihood of collisions,
but the trade-off was too computationally expensive for pre-processing.
