% Give a brief overview of what the project is and what the intended outcomes are.
% Emphasise the potential benefits of the project with quantifiable information.
% We should frame this as a problem to be solved

\section{Outline of the issue, opportunity and the business problem to be solved}

The BBC produces and stores vast amounts of data for its content, surfaced by countless services and APIs.
One of the top priorities for the BBC is to increase the usage across the business of \textbf{Passport} \cite{BBC:PassportMetadata},
an internal BBC system that provides a rich set of metadata annotations for multimodal content (audio, video and text).
The usage in production is low, and its BBC-wide adoption would make access to metadata consistent, remove duplications, and reduce effort and costs.

In addition, the current C2C solution is suboptimal and limits the quality of the generated item embeddings.
The similarity score is directly proportional to the number of values in common between any pairs of items
on a per-feature basis. However, the commonality is calculated with exact string equality, ignoring any relationship between different
categorical values expressing a similar concept (e.g. ``comedy'' and ``stand-up comedy'').
Moreover, the limited number and type of tags cannot adequately describe the content.
At the same time, the skewedness of the data distribution and a lack of pre-processing, shifts the recommendations towards the most popular categories.
Lastly, each similarity score is multiplied by a hardcoded weight that modulates the importance of a feature, but it doesn't solve
the polarising effect of a skewed distribution. Unfortunately, because these are hyperparameters and not learned weights, the model can't improve
its performances by minimising them against a cost function.
\\ \\
I identified three main opportunities for improvement that can help the BBC meet its goals:
enhancing the quality of recommendations, reducing costs, and accelerating innovation,
thus increasing the licence fee value for money. These opportunities are:

\begin{itemize}
  \item \textbf{Improve the quality of the content similarity recommendations}.
  This solution ingested a rich set of metadata that better described the content
  and applied a novel technique that improved the descriptive power of the transformed metadata,
  which improved the quality of the similarity calculation.
  \item \textbf{Build a foundational item-embeddings generator for upstream recommenders}.
  This project provided an immediate solution for non-personalised C2C recommendations that solely rely on content metadata.
  It also provided a foundational approach for embedding generation for upstream personalised recommenders,
  which could improve the quality of the personalised recommendation and increase user engagement.
  \item \textbf{Reduce costs by building a general solution for the wider BBC}.
  The pipeline ingested a dataset of content metadata composed of tags used to annotate all BBC content,
  making this solution general and reducing duplications and costs.
\end{itemize}
