% Give a brief overview of what the project is and what the intended outcomes are.
% Emphasise the potential benefits of the project with quantifiable information.
% We should frame this as a problem to be solved

\section{Outline of the issue or opportunity and the business problem to be solved}

The BBC produces and stores a vast amount of data for its content, and this data is produced and surfaced by countless services and APIs.
One of the priority for the BBC is to increase the usage across the business of \textbf{Passport} \cite{BBC:PassportMetadata}, an internal BBC
system that provides a richer set of metadata annotations for multi modal content (audio, video and text).
The usage in production is very low, and its BBC-wide adoption would make the access to metadata consistent,
removing duplications and reducing effort and costs.

Furthermore, the similarity score of the current C2C recommender is directly proportional to the number of values in common between any pairs of items
on a per-feature basis. But the commonality is calculated with an exact string equality, ignoring any relationship between different
categorical values expressing a similar concept (e.g. ``comedy'' and ``stand-up comedy'').

Moreover, the number and types of tags are not enough to sufficiently
describe the content, while the data distribution is severely skewed towards the most popular annotations, and no pre-processing
is applied.

Lastly, each similarity score is multiplied by a hardcoded weight that modulates the importance of a feature, but it doesn't solve
the polarising effect of a skewed distribution. Unfortunately, because these are hyperparameters and not learned weights, the model can't improve
its performances by minimising them against a cost function.
\\ \\
To address these issues, the aim of this project was:

\begin{itemize}
  \item \textbf{To improve the quality of the C2C similarity recommendations}. The hypothesis was that by using in input a richer set of metadata that
  better describes the content, and by reducing the high-dimensional data to a lower-dimensional latent manifold,
  the model would be able to generate embeddings that could improve the quality of the recommendations, by mapping the
  item similarity problem to a geometric distance calculation between vectors in a multi-dimensional Euclidean space.
  \item \textbf{To build a general solution that can be applied to multimodal content, reducing the costs}. I built a C2C solution for
  iPlayer, but I used Passport tags to make it general, so that it could be applied to any BBC content, because they all share the same set of common tags.
  \item \textbf{To build a foundational item-embeddings generator}. Content-based recommenders
  use item metadata. This project provided an immediate solution for non-personalised C2C recommendations that solely relies on them.
  It also provided a foundational basis for personalised recommenders that could benefit from using content embeddings combined with other data like user interactions.
\end{itemize}
