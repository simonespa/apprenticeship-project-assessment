\section{Discussion and conclusions}

% These should be framed as benefits to the business with quantifiable evidence
% It is likely the results of your project will have been shared at the company,
% if they were shown in a presentation or perhaps in a dashboard state this. If positive feedback was received, include this also.
% Reconsider your stakeholders, did the project have the expected outcomes for them?
\subsection{Results}

This is a general solution that works on any content that uses Passport tags. It could be implemented as a single implementation
to serve multiple products. This will reduce effort, duplication of code and data, and as a consequence, costs.
I presented this project explaining the main intuition and showing the results with the visualisation tool.
Once to the data scientists and engineers of the team I worked with and on a second occasion,
to the stakeholders of the team that - after an internal restructuring - will be in charge to provide unpersonalised recommendations,
including C2C similarity. The feedback was positive in both cases. When I presented the project to the last team, it was mentioned
that we would need to implement this solution so that we can assess end-to-end feasability and put it in a position to be A/B tested
against the current solution in production.

% State how your results have aligned with your original objectives and measures of success
% State what recommendations have been made to the business as a result of this project
% These recommendations may include proposals for further work or alternative research or extensions to the project
% State whether or not the machine learning approach can be successful in meeting the future requirements of the business.
% If it has been deployed state considerations such as data drift and retraining that will need to be taken into account
% It has not been deployed but there is a desire to do so within the business outline and the steps needed to achieve this
\subsection{Summary of findings and recommendations}

The results are perfectly aligned with the initial objectives and measure of success set at the beginning of the project.
My recommendation is to build an initial minimum viable product (MVP) pipeline on the AWS development account.
This will allow us to break down the engineering effort, and to spot any blockers/challenges that need to be addressed
as early as possible, so that we can correct them and/or reconsider some of the assumptions ahead of the production build.
We could use the Sagemaker Pipeline to build the stages, and initially test the end-to-end solution with batch Passport data.
We would need two pipelines, one for training and one for inference, to generate the embeddings and the similarity scores.
The output could be cached, to improve performances, and if everything goes smoothly from an engineering point of view,
we could integrate with UCED to fetch realtime data, and prep the solution for A/B testing.

If this solution is viable and passes the A/B test, it could also be used to generate embeddings for
other personalised recommender that use item metadata in conjunction with user interactions and/or contextual data such as
day and time of interaction, location, device used, etc.

The project could be further expanded by exploiting the graph nature of the data using graph neural network (GNN), specifically a
graph autoencoder (GAE) to learn meaningful representation of the graph data, capturing the topological structure and the node content.
This could improve upon the current autoencoder, that relies on the same data being flattened.
This effort will require further research and prototyping.

% Outline the repercussions the execution of this project has had for yourself, your colleagues, the stakeholders and the business as a whole.
\subsection{Implications}

The repercussions of

% State some of the shortcomings of your work either as a result of limitations in the data, the overall model performance or the time available to implement the project.
\subsection{Caveats and limitations}

Data drift can cause loss of similarity information, requiring a re-training of the model.
The limitation is when new programmes with new unseen tags are added to the catalogue.
If we don't re-run the pipeline, the new tags won't be one-hot encoded, which means they will be dropped entirely from the embedding generation.

